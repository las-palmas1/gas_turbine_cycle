\documentclass[a4paper,10pt]{article}
\usepackage{mathtext}
\usepackage[T2A]{fontenc}
\usepackage[utf8]{inputenc}
\usepackage[russian]{babel}
\usepackage{amsmath}
\usepackage{amsfonts}
\usepackage{amssymb}
\usepackage{graphicx}
\usepackage[left=2cm,right=2cm,
    top=2cm,bottom=2cm,bindingoffset=0cm]{geometry}
\usepackage{color}
\usepackage{gensymb}

\usepackage{enumitem}
\setlist[enumerate]{label*=\arabic*.}

\usepackage{indentfirst}

\usepackage{titlesec}




\begin{document}

    \section{Параметры электрогенератора.}

    
\begin{enumerate}
	\item Электрогенератор: Т-16-2Р УХЛ3.1.
	\item Мощность электрогенератора: $N_{эг} = 10.0\ МВт$.
	\item КПД электрогенератора: $\eta_{эг} = 0.97$
	\item Мощность на валу электрогенератора: $N = \frac{N_{эг}}{\eta_{эг}} = 10.309\ МВт$.
\end{enumerate}


    \section{Исходные данные.}

    
\begin{enumerate}

	\item Давление окружающей среды: $p_{н} = 0.1 \cdot 10^6\ Па$.
	\item Температура окружающей среды: $T_{н} = 288\ К$.
	\item Мощность на валу нагрузки: $ N = 10.309 \cdot 10^6\ МВт $.
	\item Температура торможения после камеры сгорания: $T_г^* = 1450\ К$.
	\item Политропический КПД компрессора: $\eta^*_{к п} = 0.89 $.
	\item Политропический КПД турбины компрессора: $\eta^*_{ткп} = 0.91$.
	\item Политропический КПД силовой турбины: $\eta^*_{тсп} = 0.91$.
	\item Низшая теплота сгорания топлива (природный газ): $Q^р_н = 48.412 \cdot 10^6\ Дж/кг$.
	\item Теоретически необходимая масса воздуха: $l_0 = 16.683\ кг/кг$.

	\item Степень сохранения полного давления во входном патрубке: $\sigma_{вх} = 0.99$.
	\item Степень сохранения полного давления в выходном патрубке: $\sigma_{вых} = 0.99$.
	\item Степень сохранения полного давления в камере сгорания: $\sigma_г = 0.98$.
	\item Коэффициент полноты сгорания: $\eta_г = 0.99 $.
	\item Относительный расход на охлаждение лопаток: $g_{охл} = 0.05$.
	\item Относительный расход на прочие нужды: $g_{ут} = 0.01$.
	\item Относительный расход воздуха, возвращаемого перед силовой турбиной: $g_{воз.тс} = 0.05$.
	\item Механический КПД на валу турбины компрессора: $\eta_{м.тк} = 0.99$.
	\item Механический КПД на валу силовой турбины: $\eta_{м.тс} = 0.99$.
	\item КПД редуктора: $ \eta_р = 0.99$.
	\item Скорость на выходе из выходного устройства: $ c_{вых} = 100 $

\end{enumerate}


    \section{Расчет.}

    

\begin{enumerate}
	
	\item Показатель адиабаты из предыдущей итерации: $k_в = 1.3852$.

	\item Определим давление за входным устройством: 
	\[p_{вх}^* = \sigma_{вх} p_{н} =
	0.99 \cdot 0.1 \cdot 10^6 =
	0.099 \cdot 10^6\ Па\]

	\item Определим давление за компрессором: 
	\[p_к^* = \pi_к p_{вх}^* = 17 \cdot
							   0.099 \cdot 10^6 
	= 1.683 \cdot 10^6 \ Па\]

	\item Определим адиабатический КПД компрессора: 
	\[\eta_{к}^* = \frac{
							\pi_к ^ {\frac{k_в - 1}{k_в} - 1}
					}{
							\pi_к ^ {\frac{k_в - 1}{k_в \eta_{кп}^* - 1}}
					} = 
		\frac{
				17 ^ {\frac{
										1.3852 - 1
										}{
										1.3852
									} - 1}
		}{
				17 ^ {\frac{
										1.3852 - 1
									}{
										1.3852 \cdot 0.89 - 1
									}}
		} 
		= 0.842\]

	\item Определим температуру газа за компрессором: 
	\[T_к^* = T_н \left[
					1 + \frac{
								\pi_к^{\frac{k_в - 1}{k_в}} - 1
							}{
								\eta_к^*
						} 
			\right] = 
			288 \cdot \left[ 
						1 + \frac{
									17 ^ {\frac{1.3852 - 1}{ 1.3852 }} - 1
								}{
									0.842
							} 
						\right] = 697.97 \/\ К\]

	\item Определим уточненное значение показателя адиабаты:
	\begin{enumerate}

		\item  Средняя теплоемкость воздуха в интервале температур от 273 К до $T_н$:

		\[c_{pв\ ср}(T_н) = 1003.98\ ДЖ/(кг \cdot К) \]

		\item Средняя теплоемкость воздуха в интервале температур от 273 К до $T_к^*$:

		\[ c_{pв\ ср}(T_к^*) = 1030.9\ ДЖ/(кг \cdot К) \]

		\item Средняя теплоемкость воздуха в интервале температур от $T_н$ до $T_к^*$:

		\[c_{pв} = \frac{
		c_{pв\ ср}(T_к^*) (T_к^* - T_0) - c_{pв\ ср}(T_н)(T_н - T_0)
		}{
		T_к^* - T_н} = \]
		\[ =\frac{
		1030.9 \cdot (697.97 - 273) -
		1003.98 \cdot (288 - 273)
		}{
		697.97 - 288} =
		1008.04 \ Дж / (кг \cdot К)\]

		\item Новое значение показателя адиабаты:

		\[k_в^\prime = \frac{c_{pв}}{c_{pв} - R_в} = 
					\frac{
					1008.04
					}{
					1008.04 - 287.4} 
					= 1.386\]

	\end{enumerate}

	\item Определим погрешность определения показателя адиабаты:
	
	\[\delta = \frac{\left| k_в^\prime - k_в \right|}{k_в} \cdot 100 \% =
	\frac{
		\left| 1.386 - 1.3852 \right|
	}{
		1.3852
	} \cdot 100 \% = 
	0.0622 \% < 1 \%\]
	Точность определения показателя адиабаты воздуха находится в пределах допуска.

	\item Определим работу компрессора:

	\[L_к = c_{pв} \left( T_к^* - T_a \right) =
			1008.04 \cdot 
			\left( 697.97 - 288 \right) = 
			0.4133 \cdot 10^6 \/\ Дж/кг \]

	\item Температура газа за камерой сгорания:

	\[T_г^* = 1450 \/\ К\]

	\item Относительный расход воздуха на входе в камеру сгорания:

	\[
	g_{вх.кс} = 
	1 - g_{охл} - g_{ут} = 
	1 - 0.05 - 0.01 =
	0.94
	\]

	\item Значение коэффициента избытка воздуха из предпоследней итерации.

	\[ \alpha = 2.8856 \]

	\item Средняя теплоемкость воздуха в интервале температур от 273 К до $ T_к^* $.
	\[ c_{pв} (T_к^*)  = 1030.9\ Дж / (кг \cdot К) \]
		
	\item Средняя теплоемкость продуктов сгорания природного газа после камеры сграния.
		
	\[ c_{pг} (T_г^*, \alpha) = 1162.97\ Дж/(кг \cdot К) \]
		
	\item Средняя теплоемкость продуктов сгорания природного газа при температуре $T_0 = 288\ К$.
		
	\[ c_{pг} (T_0, \alpha) = 1039.68\ Дж/(кг \cdot К) \]
		
	\item Относительный расход топлива в камере сгорания:
		
	\[  g_т = \frac{G_т}{G_в^г} =
		\frac{
			c_{pг} \left( T_г^* \right) T_г^* -
			c_{pв} \left( T_к^* \right) T_к^*
		}{
			Q_н^р \eta_г -
			\left[
				c_{pг} \left( T_г^* \right) T_г^* -
				c_{pг} \left( T_0 \right) T_0 \right]	} =  \]
		\[=
		\frac{
			1162.97 \cdot 1450 -
			1030.9  \cdot 697.97
		}{
			48.412 \cdot 10^6 \cdot 0.99 -
			\left[
				1162.97 \cdot 1450 -
				1039.68 \cdot 288 \right]	  }
		=  0.0208
		\]
	
	\item Новое значение коэффициента избытка воздуха:
	
	\[
	\alpha^ \prime = \frac{ 1 }{ g_т l_0 }  = 
	\frac{ 1 }{ 0.0208 \cdot 16.683 } = 2.8856
	\]
	
	\item Погрешность определения коэффициента избытка воздуха:
	
	\[
	\delta = \frac{ \left|  \alpha^\prime - \alpha \right| }{ \alpha } \cdot 100 \%  =
		\frac{ \left|  2.8856 - 2.8856 \right| }{ 2.8856 } \cdot \% =
		0.001 \%
	\]
	
	\item Относительный расход газа на входе в турбину компрессора:
	
	\[
	g_{г.тк} = g_{вх.кс} \cdot ( 1 + g_т ) =
		0.94 \cdot ( 1 + 0.0208) =
		0.9595
	\]
	
	Расчет турбины компрессора состоит из двух частей. Первая часть - это определения температуры на выходе из турбины. 
	Этот расчет является итерационным и ведется до сходимости по $k_г$.  
	Вторая часть - расчет давления торможения на выходе из турбины. Этот расчет также является итерационным и 
	ведется до сходимости по $\pi_{тк}^*$. Ниже приведены последнии итерации обоих расчетов.	
	
	\item Определим удельную работу турбины компрессора:
	
	\[
	L_{тк} = \frac{ L_к }{ g_{г.тк} \eta_{м.тк} } = 
			\frac{ 0.4133 \cdot 10^6 }{ 0.9595 \cdot 0.99 } = 
			0.435 \cdot 10^6 \/\ Дж/кг
	\]
	
	\item Определим давление газа перед турбиной:
	
	\[
	p_г^* = p_к^* \sigma_г = 1.683 \cdot 0.98 = 1.6493 \cdot 10^6\ Па
	\]
	
	\item Коэффициент адиабаты из предыдущей итерации:
	
	\[
	k_г = 1.3161
	\]
	
	\item Средняя теплоемкость газа в процессе расширения в турбине при данном показателе адиабаты:
	
	\[
	c_{pг} = \frac{ k_г }{ k_г - 1 } \cdot R_г = 
			\frac{ 1.3161 }{ 1.3161 - 1 } \cdot 300.67 = 
			1251.75\ Дж / (кг \cdot К)
	\]
	
	\item Определим температуру за турбиной компрессора:
	
	\[
	T_{тк}^* = T_к^* - \frac{ L_{тк} }{ c_{pг} } = 
			697.97 - \frac{ 0.435 \cdot 10^6  }{ 1251.75 } = 
			1080.42\ К
	\]
	
	\item Опеределим уточненное значение показателя адиабаты газа.
	
	\begin{enumerate}
	
		\item Средняя удельная теплоемкость в интервале температур от 288 К до $ T_{тк}^* $:
		
		\[
		c_{pг\ ср} (T_{тк}^*) = 1122.33\ Дж / (кг \cdot К)
		\]
		
		\item Средняя удельная теплоемкость в интервале температур от 288 К до $ T_{г}^* $:
		
		\[
		c_{pг\ ср} (T_{г}^*) = 1162.97\ Дж / (кг \cdot К)
		\]
		
		\item Новое значение средней теплоемкости в интервале температуре от $ T_{тк}^* $ от $ T_{г}^* $:
		
		\[c_{pг}^\prime = \frac{
		c_{pг\ ср}(T_г^*) (T_г^* - T_0) - c_{pг\ ср}(T_{тк}^*) (T_{тк}^* - T_0)
		}{
		T_г^* - T_{тк}^*} = \]
		
		\[ =\frac{
		1162.97 \cdot (1450 - 273) - 
		1122.33 \cdot (1080.42 - 273)
		}{
		1450 - 1080.42} = 
		1182.09 \ Дж / (кг \cdot К)
		\]
		
		\item Новое значение показателя адиабаты:
		
		\[
		k_{г}^\prime = \frac{ c_{pг}^\prime }{ c_{pг}^\prime - R_г } = 
				= \frac{ 1182.09 }{ 1182.09 - 300.67 } =
				1.3161
		\]
		
		\item Погрешность определения показателя адиабаты:
		
		\[
		\delta = \frac{ \left| k_{г}^\prime - k_{г} \right| }{ k_{г} } \cdot 100 \% =
				= \frac{ \left| 1.3161 - 1.3161 \right| }{ 1.3161 } \cdot 100 \%
				= 0.0
		\]
	
	\end{enumerate}
	
	\item Определим степень понижения давления в турбине.
	
	\begin{enumerate}
		
		\item Степень понижения давления из предыдущей итерации:
		
		\[
		\pi_{тк} = 3.502
		\]
		
		\item Адиабатический КПД турбины компрессора:
		
		\[
		\eta_{тк}^* = \frac{1 - \pi_{тк} ^ 
	                   {\frac{\left(1 - k_г \right) \eta_{ткп}^*}{k_г}}
					}{
					   1 - \pi_{тк} ^ {\frac{1 - k_г}{k_г}} 
					} = 
				\frac{1 - 3.502 ^ 
	                   {\frac{\left(1 - 1.3161 \right) 0.91 }{ 1.3161 }}
					}{
					   1 - 3.502 ^ {\frac{ 1 - 1.3161 }{ 1.3161 }} 
					} = 
			0.9218
		\]	
		
		\item Новое значение степени понижения давления в турбине компрессора:
		
		\[
		\pi_{тк}^\prime = \left[ 
							1 - \frac{L_{тк}}{c_{pг} T_г^* \eta_{тк}^*}	
						\right] ^ 
							\frac{k_г}{k_г - 1} =
					\left[ 
						1 - \frac{ 
								0.435 \cdot 10^6  
							}{ 
								1182.09 \cdot 1450 \cdot 0.9218
							}	
					\right] ^ 
						\frac{ 1.3161 }{ 1.3161 - 1} =
					3.503
		\]
		
		\item Погрешность определения степени понижения давления:
		
		\[
		\delta = \frac{ \left| \pi_{тк} - \pi_{тк}^\prime \right| }{ \pi_{тк} } \cdot 100 \% =
				\frac{ 
					\left| 3.502 - 3.503 \right|
				}{ 
					3.502 
				} \cdot 100\ \% = 
				0.0276\ \% 
		\]
	
	\end{enumerate}
	
	\item Давление на выходе из турбины компрессора:
	
	\[
	p_{тк}^* = \frac{ p_г^* }{ \pi_{тк}^\prime } = \frac{ 1.6493 \cdot 10^6 }{ 3.503 } = 
		= 0.470817 \cdot 10^6\ Па
	\]
	
	\item Относительный расход газа на входе в силовую турбину:
	
	\[ g_{г.тс} = g_{г.тк} + g_{воз} = 0.9595 + 0.05 = 1.0095 \]
	
	\item Температура на выходе из силовой турбины из предыдущей итерации: $ T_{тс}^* = 761.02\ К$.

	\item Истинная теплоемкость газа при данной температуре:
	
	\[ c_{pг\ ис} (T_{тс}^*) = 1142.59\ Дж/ (кг \cdot К) \]
	
	\item Коэффициент адиабаты:
	
	\[
	k_{г\ ис} (T_{тс}^*)  = \frac{ c_{pг\ ис} }{ c_{pг\ ис} - R_г } = 
			\frac{ 1142.59 }{ 1142.59 - 300.67 } = 
			1.3571
	\]

	\item Критическая скорость звука на выходе из выходного устройства:

	\[
		a_{кр\ вых} = \sqrt{\frac{2 k_{г\ ис}}{k_{г\ ис} + 1} \cdot R_г T_{тс}^* } =
		\sqrt{
			\frac{2 \cdot k1.3571
			}{
			1.3571 + 1} \cdot
			300.67 \cdot 761.02
		} =
		513.12\ м/с
	\]

	\item Приведенная скорость на выходе из выходного устройства:

	\[
		\lambda_{вых} = \frac{c_{вых}}{a_{кр\ вых}} =
			\frac{100}{513.12} =
		0.195
	\]

	\item Давление торможения на выходе из выходного устройства
	
	\begin{gather*}
	    p_{вых}^* = \frac{ p_н
				}{
					\left(
						1 - \frac{ k_{г\ ис} - 1 }{ k_{г\ ис} + 1 } \cdot \lambda_{вых} ^ 2
					\right)
						^ {
							\frac{ k_{г\ ис} }{ k_{г\ ис} - 1 }
						}
				} =\\
	    = \frac{ 0.1 \cdot 10^6
		}{
			\left(
				1 - \frac{ 1.3571 - 1 }{ 1.3571 + 1 } \cdot  ^ 2
			\right)
				^ {
					\frac{ 1.3571 }{ 1.3571 - 1 }
				}
				} =
		0.1022 \cdot 10^6\ Па\\
	\end{gather*}
	
	\item Определим давление торможения за силовой турбиной:
	
	\[
	p_{ст}^* = \frac{ p_{вых}^* }{ \sigma_{вых} } = \frac{ 0.1022 \cdot 10^6 }{ 0.99 } = 
			0.1032 \cdot 10^6\ Па
	\]

	\item Степень понижения давления в силовой турбине:
	
	\[ \pi_{ст} = \frac{ p_{тк}^* }{ p_{ст}^* } =
			\frac{ 
				0.4708 \cdot 10^6 
			}{ 
				0.1032 \cdot 10^6 
			} = 
			4.56
	\]
	
	\item Коэффициент адиабаты из предыдущей итерации:
	
	\[ k_г = 1.3411 \]
	
	\item Адиабатический КПД в силовой турбине:
	
	\[
	\eta_{тс}^* = \frac{
					1 - \pi_{тс} ^ 
							{\frac{ (1 - k_г ) \eta_{тсп}^* }{ k_г }}
				}{
					1 - \pi_{тс} ^ 
							{\frac{ 1 - k_г }{ k_г }} 
				} = 
			\frac{
				1 - 4.56 ^ 
						{\frac{ (1 - 1.3411 ) \cdot 0.91 }{ 1.3411 }}
			}{
				1 - 4.56 ^ 
						{\frac{ 1 - 1.3411 }{ 1.3411 }} 
			} = 
		0.925
	\]	
	
	\item Определим температуру торможения на выходе из силовой турбины:
	
	\begin{gather*}
	    T_{тс}^* = T_{тк}^*
		\left\lbrace
			1 -
			\left[
				1 -
					\left(
						\frac{ p_{тк}^* }{ p_{тс}^* }
					\right) ^ \frac{ k_г }{ k_г - 1 }
			\right] \cdot \eta_{тс}^*
		\right\rbrace =\\
	    = 1080.42 \cdot
		\left\lbrace
			1 -
			\left[
				1 -
					\left(
						\frac{ 0.4708 \cdot 10^6 }{ 0.1032 \cdot 10^6 }
					\right) ^ \frac{ 1.3411 }{ 1.3411 - 1 }
			\right] \cdot 0.925
		\right\rbrace =
	760.46\ К\\
	\end{gather*}
	
	\item Погрешность определения температуры за силовой турбиной:
	
	\[
	\delta = \frac{ 
					\left| T_{тс}^* - T_{вых}^* \right|
				}{ 
					T_{вых}^*
				} \cdot 100 \%= 
		\frac{ 
			\left| 760.46 - 761.02 \right|
		}{ 
			760.46
		} \cdot 100 \% =
	0.074 \%
	\]
	
	\item Опеределим уточненное значение показателя адиабаты газа.
	
	\begin{enumerate}
	
		\item Средняя удельная теплоемкость в интервале температур от 288 К до $ T_{тк}^* $:
		
		\[
		c_{pг\ ср} (T_{тк}^*) = 1122.33\ Дж / (кг \cdot К)
		\]
		
		\item Средняя удельная теплоемкость в интервале температур от 288 К до $ T_{тс}^* $:
		
		\[
		c_{pг\ ср} (T_{тс}^*) = 1083.1\ Дж / (кг \cdot К)
		\]
		
		\item Новое значение средней теплоемкости в интервале температуре от $ T_{тc}^* $ от $ T_{тк}^* $:
		
		\begin{gather*}
		    c_{pг}^\prime = \frac{
			c_{pг\ ср}(T_{тк}^*) (T_{тк}^* - T_0) - c_{pг\ ср}(T_{тс}^*) (T_{тс}^* - T_0)
		}{
			T_{тк}^* - T_{тс}^*} =\\
		    = \frac{
			1122.33 \cdot (1080.42 - 273) -
			1083.1 \cdot (760.46 - 273)
		}{
			1080.42 - 760.46} =
			1182.09 \ Дж / (кг \cdot К)\\
		\end{gather*}
		
		\item Новое значение показателя адиабаты:
		
		\[
		k_{г}^\prime = \frac{ c_{pг}^\prime }{ c_{pг}^\prime - R_г } = 
				= \frac{ 1182.09 }{ 1182.09 - 300.67} =
				1.3411
		\]
		
		\item Погрешность определения показателя адиабаты:
		
		\[
		\delta = \frac{ \left| k_{г}^\prime - k_{г} \right| }{ k_{г} } \cdot 100 \% =
				\frac{ \left|  1.3411 - 1.3411 \right| }{ 1.3411 } \cdot 100 \% =
				0.0021 \%
		\]
	
	\end{enumerate}
	
	\item Определим значение теплоемкости газа в свободной турбине:
	
	\[
	c_{pг} = \frac{ k_г^\prime }{ k_г^\prime - 1 } \cdot R_г = 
			\frac{ 1.3411 }{ 1.3411 - 1 } \cdot 300.67
			= 1182.09\ Дж/(кг \cdot К)
	\]
	
	\item Определим удельную работу силовой турбины:
	
	\[
	L_{тс} = c_{pг} ( T_{тк}^* -  T_{тс}^*) = 
		1182.09 \cdot ( 1080.42 -  760.46 ) = 
		0.3782 \cdot 10^6\ Дж/кг
	\]
	
	\item Определим удельную мощность ГТД:
	
	
	
	\[
	N_{e\ уд} = L_{тс} g_{г.тс} \eta_{м.тс} \eta_р = 
			0.3782 \cdot 10^6 \cdot 1.0095 \cdot 0.99 \cdot 0.99 =
	0.3742 \cdot 10^6 Дж/кг
	\]
	
	\item Определим экономичность ГТД:
	
	
	
	\[
	C_e = \frac{ 3600 }{ N_{e уд} } g_т g_{вх.кс} = 
			\frac{ 3600 }{ 0.3742 \cdot 10^6} \cdot 0.0208 \cdot 0.94 =
	0.1878 \cdot 10^{-3}\ кг/\left( Вт \cdot ч \right)
	\]
	
	\item Определим КПД ГТД:
	
	\[
	\eta_e = \frac{ 3600 }{ C_e Q_н^р } = 
			\frac{ 3600 }{ 0.1878 \cdot 10^{-3} \cdot 48.412 \cdot 10^6} 
	= 0.3959
	\]
	
	\item Определим расход воздуха:
	
	\[
	G_в = \frac{N_e}{N_{e уд} } = 
	\frac{ 10.309278350515465 \cdot 10^6 }{ 0.3742 \cdot 10^6 } = 
	27.548\ кг/с
	\]

	\item Расход топлива:

	\[
		G_{т} = g_т g_{вх.кс} G_в = 0.0208 \cdot 0.94
		\cdot 27.548 =
		0.538\ кг/с
	\]

\end{enumerate}




    \section{Параметры дожимного компрессора.}

    
\begin{enumerate}
    \item Дожимной компрессор: ТАКАТ-9/13-33,5.
    \item Средняя теплоемкость природного газа: $с_{p\ пг.ср} = 2300.0\ Дж/(кг \cdot К) $.
    \item Средний показатель адиабаты: $k_{пг.ср} = 1.31$.
    \item Плотность по ГСССД 160-93 при давлениее $p_{вх}$: $\rho_{пг} = 9.15\ кг/м^3$.
    \item Адиабатический КПД компрессора: $\eta_{ад} = 0.82$.
    \item КПД электродвигателя: $\eta_{эд} = 0.95$
    \item Массовый расход: $G_{г} = G_{т} = 0.538$.
    \item Температура на входе: $T_{вх} = 288\ К$.
    \item Начальное давление: $p_{вх} = 1.3\ МПа$.
    \item Давление на выходе: $p_{вых} = 2.183\ МПа$.
    \item Степень повышения давления:
    \[
        \pi = \frac{p_{вых}}{p_{вх}} = \frac{2.183}
        {1.3} =
        1.679
    \]
    \item Температура на выходе:
    \[
        T_{вых} = T_{вх} \cdot \left[
                1 + \frac{
                        \pi^{\frac{k_{пг.ср} - 1}{k_{пг.ср}}} - 1
                    }{ \eta_{ад} }
        \right] =
    \]
    \[
        = 288 \cdot \left[
                1 + \frac{
                        1.679 ^
                        {\frac{1.31 - 1}{1.31}} - 1
                    }{ 0.82 }
        \right] =
        333.83\ К
    \]
    \item Удельная работа:
    \[
        L_e = с_{p\ пг.ср} \cdot ( T_{вых} - T_{вх} ) =
                2300.0 \cdot ( 333.83 - 288 ) =
        105.42\ КДж/кг
    \]
    \item Мощность электродвигателя для привода компрессора:
    \[
        N_{эл} = \frac{L_e}{G_{г} \cdot \eta_{эд}} =
        \frac{105.42}
        {0.538 \cdot 0.95 } =
        206.29\ КВт.
    \]
    \item Электрическая мощность за вычетом затрат на привод компрессора: $N_{эл} = N_{эг} - N_{к} = 9.79$ МВт.
    \item Производительность компрессора:
    \[
        Q = \frac{60 \cdot G_г}{\rho_{пг}} = \frac{60 \cdot 0.538}
        {9.15} = 3.528\ м^{3}/мин.
    \]
\end{enumerate}




\end{document}
\documentclass[a4paper,10pt]{article}
\usepackage{mathtext}
\usepackage[T2A]{fontenc}
\usepackage[utf8]{inputenc}
\usepackage[russian]{babel}
\usepackage{amsmath}
\usepackage{amsfonts}
\usepackage{amssymb}
\usepackage{graphicx}
\usepackage[left=2cm,right=2cm,
    top=2cm,bottom=2cm,bindingoffset=0cm]{geometry}
\usepackage{color}
\usepackage{gensymb}

\usepackage{enumitem}
\setlist[enumerate]{label*=\arabic*.}

\usepackage{indentfirst}

\usepackage{titlesec}



\begin{document}




\section{Расчет цикла для выбранного $\pi_к^*$.}

\subsection{Исходные данные.}

\begin{enumerate}

	\item Давление окружающей среды: $p_{н} = 0.1 \cdot 10^6\ Па$.
	\item Температура окружающей среды: $T_{н} = 288\ К$.
	\item Степень повышения давления по параметрам торможения в компрессоре: $\pi_к^*= 10$.
	\item Мощность на валу силовой турбины: $ N = 2.0 \cdot 10^6\ МВт $.
	\item Температура торможения после камеры сгорания: $T_г^* = 1400\ К$.
	\item Политропический КПД компрессора: $\eta^*_{к п} = 0.89 $.
	\item Политропический КПД турбины компрессора: $\eta^*_{ткп} = 0.91$.
	\item Политропический КПД силовой турбины: $\eta^*_{тсп} = 0.91$.
	\item Низшая теплота сгорания топлива (риродный газ): $Q^р_н = 48.412 \cdot 10^6\ Дж/кг$.
	\item Теоретически необходимая масса воздуха: $l_0 = 16.683\ кг/кг$.

	\item Степень сохранения полного давления во входном патрубке: $\sigma_{вх} = 0.99$.
	\item Степень сохранения полного давления в выходном патрубке: $\sigma_{вых} = 0.99$.
	\item Степень сохранения полного давления в камере сгорания: $\sigma_г = 0.98$.
	\item Коэффициент полноты сгорания: $\eta_г = 0.99 $. 
	\item Относительный расход на охлаждение лопаток: $g_{охл} = 0.04$.
	\item Относительный расход на прочие нужды: $g_{ут} = 0.01$.
	\item Относительный расход воздуха, возвращаемого перед турбиной компрессора: $g_{воз.тк} = 0.01$.
	\item Относительный расход воздуха, возвращаемого перед силовой турбиной: $g_{воз.тс} = 0.01$.
	\item Механический КПД на валу турбины компрессора: $\eta_{м.тк} = 0.99$.
	\item Механический КПД на валу силовой турбины: $\eta_{м.тс} = 0.99$.
	\item КПД редуктора: $ \eta_р = 0.99$.
	\item Приведенная скорость на выходе из выходного устройства: $ \lambda_{вых} = 0.04 $

\end{enumerate}

\subsection{Расчет.}

\begin{enumerate}
	
	\item Показатель адиабаты из предыдущей итерации: $k_в = 1.3946$.

	\item Определим давление за входным устройством: 
	\[p_{вх}^* = \sigma_{вх} p_{н} =
	0.99 \cdot 0.1 \cdot 10^6 = 
	0.099 \cdot 10^6\ Па\]

	\item Определим давление за компрессором: 
	\[p_к^* = \pi_к p_{вх}^* = 10 \cdot
							   0.099 \cdot 10^6 
	= 0.99 \cdot 10^6 \ Па\]

	\item Определим адиабатический КПД компрессора: 
	\[\eta_{к}^* = \frac{
							\pi_к ^ {\frac{k_в - 1}{k_в} - 1}
					}{
							\pi_к ^ {\frac{k_в - 1}{k_в \eta_{кп}^* - 1}}
					} = 
		\frac{
				10 ^ {\frac{
										1.3946 - 1
										}{
										1.3946
									} - 1}
		}{
				10 ^ {\frac{
										1.3946 - 1
									}{
										1.3946 \cdot 0.89 - 1
									}}
		} 
		= 0.8509\]

	\item Определим температуру газа за компрессором: 
	\[T_к^* = T_н \left[
					1 + \frac{
								\pi_к^{\frac{k_в - 1}{k_в}} - 1
							}{
								\eta_к^*
						} 
			\right] = 
			288 \cdot \left[ 
						1 + \frac{
									10 ^ {\frac{1.3946 - 1}{ 1.3946 }} - 1
								}{
									0.8509
							} 
						\right] = 598.86 \/\ К\]

	\item Определим уточненное значение показателя адиабаты:
	\begin{enumerate}

		\item  Средняя теплоемкость воздуха в интервале температур от 273 К до $T_н$:

		\[c_{pв\ ср}(T_н) = 999.09\ ДЖ/(кг \cdot К) \]

		\item Средняя теплоемкость воздуха в интервале температур от 273 К до $T_к^*$:

		\[ c_{pв\ ср}(T_к^*) = 1014.7\ ДЖ/(кг \cdot К) \]

		\item Средняя теплоемкость воздуха в интервале температур от $T_н$ до $T_к^*$:

		\[c_{pв} = \frac{
		c_{pв\ ср}(T_к^*) (T_к^* - T_0) - c_{pв\ ср}(T_н)(T_н - T_0)
		}{
		T_к^* - T_a} = \]
		\[ =\frac{
		1014.7 \cdot (598.86 - 273) -
		999.09 \cdot (288 - 273)
		}{
		598.86 - 288} =
		1005.47 \ Дж / (кг \cdot К)\]

		\item Новое значение показателя адиабаты:

		\[k_в^\prime = \frac{c_{pв}}{c_{pв} - R_в} = 
					\frac{
					1005.47
					}{
					1005.47 - 287.4} 
					= 1.3947\]

	\end{enumerate}

	\item Определим погрешность определения показателя адиабаты:
	
	\[\delta = \frac{\left| k_в^\prime - k_в \right|}{k_в} \cdot 100 \% =
	\frac{
		\left| 1.3947 - 1.3946 \right|
	}{
		1.3946
	} \cdot 100 \% = 
	0.0087 \% < 1 \%\]
	Точность определения показателя адиабаты воздуха находится в пределах допуска.

	\item Определим работу компрессора:

	\[L_к = c_{pв} \left( T_к^* - T_a \right) =
			1005.47 \cdot 
			\left( 598.86 - 288 \right) = 
			0.3126 \cdot 10^6 \/\ Дж/кг \]

	\item Температура газа за камерой сгорания:

	\[T_г^* = 1400 \/\ К\]

	\item Относительный расход воздуха на входе в камеру сгорания:

	\[
	g_{вх.кс} = 
	1 - g_{охл} - g_{ут} = 
	1 - 0.04 - 0.01 =
	0.95
	\]

	\item Значение коэффициента избытка воздуха из предпоследней итерации.

	\[ \alpha = 2.7425 \]

	\item Средняя теплоемкость воздуха в интервале температур от 273 К до $ T_к^* $.
	\[ c_{pв} (T_к^*)  = 1014.7\ Дж / (кг \cdot К) \]
		
	\item Средняя теплоемкость продуктов сгорания природного газа после камеры сграния.
		
	\[ c_{pг} (T_г^*, \alpha) = 1161.59\ Дж/(кг \cdot К) \]
		
	\item Средняя теплоемкость продуктов сгорания природного газа при температуре $T_0 = 288\ К$.
		
	\[ c_{pг} (T_0, \alpha) = 1042.02\ Дж/(кг \cdot К) \]
		
	\item Относительный расход топлива в камере сгорания:
		
	\[  g_т = \frac{G_т}{G_в^г} = 
		\frac{
			c_{pг} \left( T_г^* \right) T_г^* - 
			c_{pв} \left( T_к^* \right) T_к^* 
		}{
			Q_н^р \eta_г - 
			\left[
				c_{pг} \left( T_г^* \right) T_г^* - 
				c_{pг} \left( T_0 \right) T_0 \right]	} =  \]
		\[= 
		\frac{
			1161.59 \cdot 1400 -
			1014.7  \cdot 598.86
		}{
			48.412 \cdot 10^6 \cdot 0.99 -
			\left[
				1161.59 \cdot 1400 -
				1042.02 \cdot 288 \right]	  }
		=  0.0219
		\]
	
	\item Новое значение коэффициента избытка воздуха:
	
	\[
	\alpha^ \prime = \frac{ 1 }{ g_т l_0 }  = 
	\frac{ 1 }{ 0.0219 \cdot 16.683 } = 2.7425
	\]
	
	\item Погрешность определения коэффициента избытка воздуха:
	
	\[
	\delta = \frac{ \left|  \alpha^\prime - \alpha \right| }{ \alpha } \cdot 100 \%  =
		\frac{ \left|  2.7425 - 2.7425 \right| }{ 2.7425 } \cdot \% =
		0.0001 \%
	\]
	
	\item Относительный расход газа на входе в турбину компрессора:
	
	\[
	g_{г.тк} = g_{вх.кс} \cdot ( 1 + g_т ) + g_{воз.тк} = 
		0.95 \cdot ( 1 + 0.0219) + 0.01 = 
		0.9808
	\]
	
	Расчет турбины компрессора состоит из двух частей. Первая часть - это определения температуры на выходе из турбины. 
	Этот расчет является итерационным и ведется до сходимости по $k_г$.  
	Вторая часть - расчет давления торможения на выходе из турбины. Этот расчет также является итерационным и 
	ведется до сходимости по $\pi_{тк}^*$. Ниже приведены последнии итерации обоих расчетов.	
	
	\item Определим удельную работу турбины компрессора:
	
	\[
	L_{тк} = \frac{ L_к }{ g_{г.тк} \eta_{м.тк} } = 
			\frac{ 0.3126 \cdot 10^6 }{ 0.9808 \cdot 0.99 } = 
			0.3219 \cdot 10^6 \/\ Дж/кг
	\]
	
	\item Определим давление газа перед турбиной:
	
	\[
	p_г^* = p_к^* \sigma_г = 0.99 \cdot 0.98 = 0.9702 \cdot 10^6\ Па
	\]
	
	\item Коэффициент адиабаты из предыдущей итерации:
	
	\[
	k_г = 1.3145
	\]
	
	\item Средняя теплоемкость газа в процессе расширения в турбине при данном показателе адиабаты:
	
	\[
	c_{pг} = \frac{ k_г }{ k_г - 1 } \cdot R_г = 
			\frac{ 1.3145 }{ 1.3145 - 1 } \cdot 300.67 = 
			1256.62\ Дж / (кг \cdot К)
	\]
	
	\item Определим температуру за турбиной компрессора:
	
	\[
	T_{тк}^* = T_к^* - \frac{ L_{тк} }{ c_{pг} } = 
			598.86 - \frac{ 0.3219 \cdot 10^6  }{ 1256.62 } = 
			1126.26\ К
	\]
	
	\item Опеределим уточненное значение показателя адиабаты газа.
	
	\begin{enumerate}
	
		\item Средняя удельная теплоемкость в интервале температур от 288 К до $ T_{тк}^* $:
		
		\[
		c_{pг\ ср} (T_{тк}^*) = 1131.1\ Дж / (кг \cdot К)
		\]
		
		\item Средняя удельная теплоемкость в интервале температур от 288 К до $ T_{г}^* $:
		
		\[
		c_{pг\ ср} (T_{г}^*) = 1161.59\ Дж / (кг \cdot К)
		\]
		
		\item Новое значение средней теплоемкости в интервале температуре от $ T_{тк}^* $ от $ T_{г}^* $:
		
		\[c_{pг}^\prime = \frac{
		c_{pг\ ср}(T_г^*) (T_г^* - T_0) - c_{pг\ ср}(T_{тк}^*) (T_{тк}^* - T_0)
		}{
		T_г^* - T_{тк}^*} = \]
		
		\[ =\frac{
		1161.59 \cdot (1400 - 273) - 
		1131.1 \cdot (1126.26 - 273)
		}{
		1400 - 1126.26} = 
		1199.61 \ Дж / (кг \cdot К)
		\]
		
		\item Новое значение показателя адиабаты:
		
		\[
		k_{г}^\prime = \frac{ c_{pг}^\prime }{ c_{pг}^\prime - R_г } = 
				= \frac{ 1199.61 }{ 1199.61 - 300.67 } =
				1.3145
		\]
		
		\item Погрешность определения показателя адиабаты:
		
		\[
		\delta = \frac{ \left| k_{г}^\prime - k_{г} \right| }{ k_{г} } \cdot 100 \% =
				= \frac{ \left| 1.3145 - 1.3145 \right| }{ 1.3145 } \cdot 100 \%
				= 0.0
		\]
	
	\end{enumerate}
	
	\item Определим степень понижения давления в турбине.
	
	\begin{enumerate}
		
		\item Степень понижения давления из предыдущей итерации:
		
		\[
		\pi_{тк} = 2.53
		\]
		
		\item Адиабатический КПД турбины компрессора:
		
		\[
		\eta_{тк}^* = \frac{1 - \pi_{тк} ^ 
	                   {\frac{\left(1 - k_г \right) \eta_{ткп}^*}{k_г}}
					}{
					   1 - \pi_{тк} ^ {\frac{1 - k_г}{k_г}} 
					} = 
				\frac{1 - 2.53 ^ 
	                   {\frac{\left(1 - 1.3145 \right) 0.91 }{ 1.3145 }}
					}{
					   1 - 2.53 ^ {\frac{ 1 - 1.3145 }{ 1.3145 }} 
					} = 
			0.9188
		\]	
		
		\item Новое значение степени понижения давления в турбине компрессора:
		
		\[
		\pi_{тк}^\prime = \left[ 
							1 - \frac{L_{тк}}{c_{pг} T_г^* \eta_{тк}^*}	
						\right] ^ 
							\frac{k_г}{k_г - 1} =
					\left[ 
						1 - \frac{ 
								0.3219 \cdot 10^6  
							}{ 
								1199.61 \cdot 1400 \cdot 0.9188
							}	
					\right] ^ 
						\frac{ 1.3145 }{ 1.3145 - 1} =
					2.53
		\]
		
		\item Погрешность определения степени понижения давления:
		
		\[
		\delta = \frac{ \left| \pi_{тк} - \pi_{тк}^\prime \right| }{ \pi_{тк} } \cdot 100 \% =
				\frac{ 
					\left| 2.53 - 2.53 \right|
				}{ 
					2.53 
				} \cdot 100\ \% = 
				0.0106\ \% 
		\]
	
	\end{enumerate}
	
	\item Давление на выходе из турбины компрессора:
	
	\[
	p_{тк}^* = \frac{ p_г^* }{ \pi_{тк}^\prime } = \frac{ 0.9702 \cdot 10^6 }{ 2.53 } = 
		= 0.3835 \cdot 10^6\ Па
	\]
	
	\item Относительный расход газа на входе в силовую турбину:
	
	\[ g_{г.тс} = g_{г.тк} + g_{воз.тс} = 0.9808 + 0.01 = 0.9808 \]
	
	\item Температура на выходе из силовой турбины из предыдущей итерации: $ T_{тс}^* = 831.58\ К$.

	\item Истинная теплоемкость газа при данной температуре:
	
	\[ c_{pг\ ис} (T_{тс}^*) = 1042.91\ Дж/ (кг \cdot К) \]
	
	\item Коэффициент адиабаты:
	
	\[
	k_{г\ ис} (T_{тс}^*)  = \frac{ c_{pг\ ис} }{ c_{pг\ ис} - R_г } = 
			\frac{ 1042.91 }{ 1042.91 - 300.67 } = 
			1.4051
	\]
	
	\item Давление торможения на выходе из выходного устройства
	
	\[
	p_{вых}^* = \frac{ p_н 
				}{
					\left(  
						1 - \frac{ k_{г\ ис} - 1 }{ k_{г\ ис} + 1 } \cdot \lambda_{вых} ^ 2
					\right) 
						^ {
							\frac{ k_{г\ ис} }{ k_{г\ ис} - 1 }
						}
				} = 
	\]
	
	\[
	= \frac{ 0.1 \cdot 10^6 
		}{
			\left(  
				1 - \frac{ 1.4051 - 1 }{ 1.4051 + 1 } \cdot 0.04 ^ 2
			\right) 
				^ {
					\frac{ 1.4051 }{ 1.4051 - 1 }
				}
				} = 
		0.1001 \cdot 10^6\ Па
	\]
	
	\item Определим давление торможения за силовой турбиной:
	
	\[
	p_{ст}^* = \frac{ p_{вых}^* }{ \sigma_{вых} } = \frac{ 0.1001 \cdot 10^6 }{ 0.99 } = 
			0.1011 \cdot 10^6\ Па
	\]

	\item Степень понижения давления в силовой турбине:
	
	\[ \pi_{ст} = \frac{ p_{тк}^* }{ p_{ст}^* } =
			\frac{ 
				0.3835 \cdot 10^6 
			}{ 
				0.1011 \cdot 10^6 
			} = 
			3.793
	\]
	
	\item Коэффициент адиабаты из предыдущей итерации:
	
	\[ k_г = 1.3344 \]
	
	\item Адиабатический КПД в силовой турбине:
	
	\[
	\eta_{тс}^* = \frac{
					1 - \pi_{тс} ^ 
							{\frac{ (1 - k_г ) \eta_{тсп}^* }{ k_г }}
				}{
					1 - \pi_{тс} ^ 
							{\frac{ 1 - k_г }{ k_г }} 
				} = 
			\frac{
				1 - 3.793 ^ 
						{\frac{ (1 - 1.3344 ) \cdot 0.91 }{ 1.3344 }}
			}{
				1 - 3.793 ^ 
						{\frac{ 1 - 1.3344 }{ 1.3344 }} 
			} = 
		0.923
	\]	
	
	\item Определим температуру торможения на выходе из силовой турбины:
	
	\[
	T_{тс}^* = T_{тк}^* 
		\left\lbrace 
			1 - 
			\left[ 
				1 - 
					\left(
						\frac{ p_{тк}^* }{ p_{тс}^* }
					\right) ^ \frac{ k_г }{ k_г - 1 }
			\right] \cdot \eta_{тс}^*
		\right\rbrace = 
	\]
	\[
	= 1126.26 \cdot
		\left\lbrace 
			1 - 
			\left[ 
				1 - 
					\left(
						\frac{ 0.3835 \cdot 10^6 }{ 0.1011 \cdot 10^6 }
					\right) ^ \frac{ 1.3344 }{ 1.3344 - 1 }
			\right] \cdot 0.923
		\right\rbrace = 
	830.97\ К
	\]
	
	\item Погрешность определения температуры за силовой турбиной:
	
	\[
	\delta = \frac{ 
					\left| T_{тс}^* - T_{вых}^* \right|
				}{ 
					T_{вых}^*
				} \cdot 100 \%= 
		\frac{ 
			\left| 830.97 - 831.58 \right|
		}{ 
			830.97
		} \cdot 100 \% =
	0.074
	\]
	
	\item Опеределим уточненное значение показателя адиабаты газа.
	
	\begin{enumerate}
	
		\item Средняя удельная теплоемкость в интервале температур от 288 К до $ T_{тк}^* $:
		
		\[
		c_{pг\ ср} (T_{тк}^*) = 1131.1\ Дж / (кг \cdot К)
		\]
		
		\item Средняя удельная теплоемкость в интервале температур от 288 К до $ T_{тс}^* $:
		
		\[
		c_{pг\ ср} (T_{тс}^*) = 1094.84\ Дж / (кг \cdot К)
		\]
		
		\item Новое значение средней теплоемкости в интервале температуре от $ T_{тc}^* $ от $ T_{тк}^* $:
		
		\[
		c_{pг}^\prime = \frac{
			c_{pг\ ср}(T_{тк}^*) (T_{тк}^* - T_0) - c_{pг\ ср}(T_{тс}^*) (T_{тс}^* - T_0)
		}{
			T_{тк}^* - T_{тс}^*} = 
		\]
		
		\[ = \frac{
			1131.1 \cdot (1126.26 - 273) - 
			1094.84 \cdot (830.97 - 273)
		}{
			1126.26 - 830.97} = 
			1199.61 \ Дж / (кг \cdot К)
		\]
		
		\item Новое значение показателя адиабаты:
		
		\[
		k_{г}^\prime = \frac{ c_{pг}^\prime }{ c_{pг}^\prime - R_г } = 
				= \frac{ 1199.61 }{ 1199.61 - 300.67} =
				1.3345
		\]
		
		\item Погрешность определения показателя адиабаты:
		
		\[
		\delta = \frac{ \left| k_{г}^\prime - k_{г} \right| }{ k_{г} } \cdot 100 \% =
				\frac{ \left|  1.3345 - 1.3344 \right| }{ 1.3344 } \cdot 100 \% =
				0.002
		\]
	
	\end{enumerate}
	
	\item Определим значение теплоемкости газа в свободной турбине:
	
	\[
	c_{pг} = \frac{ k_г^\prime }{ k_г^\prime - 1 } \cdot R_г = 
			\frac{ 1.3345 }{ 1.3345 - 1 } \cdot 300.67
			= 1199.61\ Дж/(кг \cdot К)
	\]
	
	\item Определим удельную работу силовой турбины:
	
	\[
	L_{тс} = c_{pг} ( T_{тк}^* -  T_{тс}^*) = 
		1199.61 \cdot ( 1126.26 -  830.97 ) = 
		0.3542 \cdot 10^6\ Дж/кг
	\]
	
	\item Определим удельную мощность ГТД:
	
	
	
	\[
	N_{e\ уд} = L_{тс} g_{г.тс} \eta_{м.тс} \eta_р = 
			0.3542 \cdot 10^6 \cdot 0.9808 \cdot 0.99 \cdot 0.99 =
	0.3405 \cdot 10^6 Дж/кг
	\]
	
	\item Определим экономичность ГТД:
	
	
	
	\[
	C_e = \frac{ 3600 }{ N_{e уд} } g_т g_{вх.кс} = 
			\frac{ 3600 }{ 0.3405 \cdot 10^6} \cdot 0.0219 \cdot 0.95 = 
	0.2195 \cdot 10^{-3}\ кг/\left( Вт \cdot ч \right)
	\]
	
	\item Определим КПД ГТД:
	
	\[
	\eta_e = \frac{ 3600 }{ C_e Q_н^р } = 
			\frac{ 3600 }{ 0.2195 \cdot 10^{-3} \cdot 48.412 \cdot 10^6} 
	= 0.3387
	\]
	
	\item Определим расход воздуха:
	
	\[
	G_в = \frac{N_e}{N_{e уд} } = 
	\frac{ 2.0 \cdot 10^6 }{ 0.3405 \cdot 10^6 } = 
	5.874\ кг/с
	\]

\end{enumerate}




\end{document}